% !TEX root = ../main.tex
\chapter{Preface}
\addcontentsline{toc}{chapter}{Preface}
% \minitoc

% \lipsum % dummy text - remove from real document

\section{Mission}

The need for a manuscript like the one you are about to read comes
from my beliefs that there seems to exist a great disconnection
between Computer Science students (ideally the reader) and their
machines (ideally their computer). This manuscript aims to tackle that
issue by presenting essential, and well-founded, tools and
philosophies that will allow the reader to better yield
its productivity and, most importantly, have fun while using a computer.

Computer Science is fortunate enough to be a domain of Science that
has a very low maintainability. Consider for instance the Medical or
Biological fields. It would not be fair to compare these fields in how
close one can be to their instrument of work. Sleeping in a
laboratory near a microscope can be seen as weird and suspicious.
Moreover, while a more expensive and cutting-edge CT
scanner\footnote{\url{https://en.wikipedia.org/wiki/CT_scan}} can
better diagnose patients, carrying a two kilogram laptop equipped with
the latest GPU model and a keyboard capable of representing the full
rainbow will only give you back problems and headaches. From personal
experience and observation, a high-end laptop or personal computer is
bound to be, sooner or later, a pit of distraction. 

I do not want to bore you, but there is so much to cover! For that
matter, I will highlight important topics but also mark extra or auxiliary remarks.
\marginpar{\includegraphics[scale=0.05]{figure/book.png} \\\tiny important}
\marginpar{\includegraphics[scale=0.05]{figure/eye.png} \\\tiny worth checking}

This is a long endeavor and ideally never complete. I am committed to
capture your interest in your computer and in the vast ways that it
can help you. Only on the condition that you also take the time to
explore it without my help.

\section{Consider helping}

I would gladly appreciate that, if you find typos or wish to suggest new topics or changes to the
document, you would open a new issue in the repository:
\begin{itemize}
    \item GitHub Repo: \url{https://github.com/ionchirica/cs-guide}
\end{itemize}

\section{Copyright and License}
The template for this document is heavily inspired by the following work:
\begin{itemize}
    \item GitHub Repo: \url{https://github.com/Jue-Xu/Latex-Template-for-Scientific-Style-Book}
\end{itemize}
The following template, mostly for monographs, is also recommended
checking out:
\begin{itemize}
    \item GitHub Repo: \url{https://github.com/joaomlourenco/novathesis}
\end{itemize}
